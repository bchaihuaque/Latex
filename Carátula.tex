\begin{titlepage}
	\centering
		\begin{figure}[h]
		\centering
		\includegraphics[scale=0.3]{pucp.pdf}
	\end{figure}
	{\bfseries\LARGE Pontificia Universidad Católica del Perú \par}
	\vspace{0.2cm}
	{\scshape\Large Facultad de Gestión y Alta Dirección \par}
	\vspace{3cm}
	{\scshape\Large Curso: Teoría Organizacional \par}
	{\scshape\large Serie: Notas de Clase\par}	
	\vspace{1cm}
	{\scshape\large Sesión 1 \par}
	%%% https://en.wikibooks.org/wiki/LaTeX/Fonts
	
	\vspace{3cm}
	{\itshape\Large Introducción a la Teoría Organizacional\footnote{Esta obra ha sido adaptada como material de estudio para el curso Teoría Organizacional. Su uso es complementado con otros instrumentos de aprendizaje, incluyendo: casos, preguntas de discusión, dinámicas colaborativas, diapositivas de reforzamiento, instructivos pedagógicos, entre otros. Cualquier observación que tenga sobre este documento, por favor sírvase contactar al siguiente email: \href{mailto:bchaihuaque@pucp.pe}{bchaihuaque@pucp.pe}. 
			Solo se autoriza el uso y revisión de esta obra a los alumnos del curso Teoría Organizacional GES208 H 0514 Ciclo 2020-1. Prohibida la distribución y reproducción total o parcial bajo ningún medio de soporte digital o material sin consentimiento expreso del autor. © Copyright Bruno Chaihuaque Dueñas, 2020-2021
		}
		\par}
	\vspace{3cm}
	{\Large Bruno Chaihuaque-Dueñas\footnote{\url{https://orcid.org/0000-0001-5982-8505}}\par}
	{\Large Marzo 2021 \par}
\end{titlepage}
%%%% Fin Carátula